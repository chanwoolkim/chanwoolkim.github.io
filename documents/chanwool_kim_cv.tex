\documentclass{resume} % Use the custom resume.cls style

\usepackage[left=0.75in, top=0.6in, right=0.75in, bottom=0.6in]{geometry} % Document margins
\usepackage[hidelinks]{hyperref}
\usepackage{lipsum}
%\usepackage{fontspec}
%\setmonofont{Courier}
\newcommand{\tab}[1]{\hspace{.2667\textwidth}\rlap{#1}}
\newcommand{\itab}[1]{\hspace{0em}\rlap{#1}}
\name{Chanwool Kim}

\begin{document}
	\vspace{-8mm}
	\begin{center}
	\href{mailto:chanwoolkim@uchicago.edu}{chanwoolkim@uchicago.edu} -- \href{chanwoolkim.github.io}{chanwoolkim.github.io} -- (312) 961-2785
	\end{center}
	\vspace{2mm}
	\begin{minipage}[t]{0.32\textwidth}
		\textbf{Placement Directors:} \\
		\\
		\textbf{Graduate Administrator:} \\
	\end{minipage}
	\begin{minipage}[t]{0.23\textwidth}
		Ufuk Akcigit \\
		Manasi Deshpande \\
		Kathryn Falzareno \\
	\end{minipage}
	\begin{minipage}[t]{0.29\textwidth}
		\href{mailto:uakcigit@uchicago.edu}{uakcigit@uchicago.edu} \\
		\href{mailto:mdeshpande@uchicago.edu}{mdeshpande@uchicago.edu} \\
		\href{mailto:kfalzareno@uchicago.edu}{kfalzareno@uchicago.edu} \\
	\end{minipage}
	\begin{minipage}[t]{0.15\textwidth}
		(773) 702 0433 \\
		(773) 702-8260 \\
		(773) 702-3026 \\
	\end{minipage}
	
	\begin{rSection}{Office Contact Information}
		University of Chicago \\
		Kenneth C. Griffin Department of Economics \\
		Saieh Hall for Economics \\
		5757 S University Ave \\
		Chicago, IL 60637
	\end{rSection}
	
	%----------------------------------------------------------------------------------------
	%	EDUCATION SECTION
	%----------------------------------------------------------------------------------------
	\begin{rSection}{Education}
		\textbf{University of Chicago}, Ph.D. Economics \hfill {\em Expected 2026} \\
		\textbf{\textit{Ibid.}}, M.A. Economics \hfill {\em 2019} \\
		\textbf{\textit{Ibid.}}, B.A. Economics (Honors), Public Policy Studies (Honors), Statistics \hfill {\em 2017} \\
		\textbf{\textit{Ibid.}}, B.S. Mathematics \hfill {\em 2017}
	\end{rSection}
	
	%----------------------------------------------------------------------------------------
	%	REFERENCES
	%----------------------------------------------------------------------------------------
	
	\begin{rSection}{References}	
		\begin{minipage}[t]{0.5\textwidth}
			Professor Jean-Pierre Dub\'e \\ %(Chair)%
			University of Chicago \\
			Booth School of Business \\
			\href{mailto:jdube@chicagobooth.edu}{jdube@chicagobooth.edu}\\
			(773) 834-5377 \\
			
			Professor Milena Almagro \\
			University of Chicago \\
			Booth School of Business \\
			\href{mailto:milena.almagro@chicagobooth.edu}{milena.almagro@chicagobooth.edu}\\
			(773) 702-7743
		\end{minipage}
		\begin{minipage}[t]{0.5\textwidth}
			Professor Ali Horta\c{c}su \\
			University of Chicago \\
			Department of Economics \\
			\href{mailto:hortacsu@uchicago.edu}{hortacsu@uchicago.edu}\\
			(773) 702-5841
		\end{minipage}
	\end{rSection}

	%----------------------------------------------------------------------------------------
	%	TEACHING AND RESEARCH FIELDS
	%----------------------------------------------------------------------------------------
	\begin{rSection}{Research and Teaching Fields}
		\begin{minipage}[t]{0.125\textwidth}
			Primary: \\
			Secondary:
		\end{minipage}
		\begin{minipage}[t]{0.875\textwidth}
			industrial organization, spatial economics \\
			public economics, health economics, household finance
		\end{minipage}
	\end{rSection}

	%----------------------------------------------------------------------------------------
	%	JOB MARKET PAPER
	%----------------------------------------------------------------------------------------
	\begin{rSection}{Job Market Paper}
		\href{chanwoolkim.github.io}{\textbf{Retail Drugstore Closures and the Declining Drug-Retail Complementarity}}
		
		Abstract: \textit{State and local governments have proposed policies to improve pharmacy margins, hoping they will reverse a recent wave of retail drugstore closures and the alleged proliferation of pharmacy deserts in disadvantaged neighborhoods. I study the U.S. pharmacy market structure using a novel database that tracks both drugstore sales, prices, and margins, as well as consumer shopping behavior. I show that the market comprises two dominant business models: large, multi-product chain-store pharmacies (``chains'') that sell both prescription drugs (Rx) and consumer packaged goods (CPG), and small, independent, pure-play pharmacies (``independents''). The exit rate since 2015 has been driven primarily by the erosion of the chains’ business model that uses low-margin Rx as loss leaders for CPG purchases. I document four stylized empirical facts: (i) pharmacy exits are concentrated primarily among chains, as opposed to independents; (ii) exits harm local consumers by temporarily depressing drug compliance; (iii) but exits occur disproportionately in already unconcentrated markets, with no expansion in the prevalence of ``pharmacy deserts''; and (iv) exit rates appear to be correlated more with declines in trips involving CPG purchases than declines in Rx margins. To evaluate the impact of proposed Rx price reform policies, I conduct counterfactual simulations using a dynamic oligopoly model of pharmacy entry, format, location, and CPG pricing while payers set uniform Rx reimbursement rates. I find that restoring pre-decline Rx margins, as in 2015, yields only modest increases in store counts, with negligible improvements in compliance for low-income households. In contrast, directly subsidizing drug prices that consumers bear would significantly improve compliance. These findings counter the argument that policies aimed at reducing health inequality by improving drug compliance should prioritize the pharmacy channel and hence rely on boosting drug profitability as a primary lever.}
	\end{rSection}

	%----------------------------------------------------------------------------------------
	%	PUBLICATIONS
	%----------------------------------------------------------------------------------------
	% \begin{rSection}{Publications}
	%	\href{www.personalwebsite.com/research/pub1.pdf}{\textbf{Title of Publication 1}} (with First Last), \textit{Journal Name}, Other Info
		
	%	Abstract: \textit{\lipsum[4]}\\
		
	%	\href{www.personalwebsite.com/research/pub2.pdf}{\textbf{Title of Publication 2}} (with First Last), \textit{Journal Name}, Other Info
		
	%	Abstract: \textit{\lipsum[6]}
	% \end{rSection}

	%----------------------------------------------------------------------------------------
	%	WORKING PAPERS
	%----------------------------------------------------------------------------------------
	\begin{rSection}{Working Papers}
		\href{chanwoolkim.github.io}{\textbf{Speculative Demand Displacement: Evidence from the Korean Housing Market}}
		
		Abstract: \textit{This paper argues that an area-specific housing policy that aims to discourage speculators may lead to demand spillovers and increase housing prices in nearby areas. I use Korean administrative transactions, real estate registration, and large online platform data and leverage on a policy that required owner-occupancy for new home purchases. To measure the price changes after the policy, I estimate an empirical model that embeds a difference-in-differences design that compares the regions near the boundaries of the policy-applied neighborhoods, yielding several new results. First, owner-occupancy requirements indeed had significant price suppression effects in the treated areas compared to the untreated by about 6\%. Second, the price increased more than the overall increase in the city in the nearby untreated areas, suggesting that speculative demand shifted towards the untreated areas with similar neighborhood characteristics. To quantify this mechanism, I set up a neighborhood sorting model of owner-occupant households and investors making housing purchase decisions, along with quantification plans that include the designation of speculative investors and structural analysis.}
		
		\href{chanwoolkim.github.io}{\textbf{Living Standard and Psychological-Wealth-Based Optimal Policies}} \\ (with Seyoung Park and Yonghyun Shin)
		
		Abstract: \textit{We develop a new dynamic continuous-time model of optimal consumption and savings with endogenous liquidity constraints. In addition to exogenously imposed liquidity constraints, we endogenize a liquidity constraint over which individuals can maintain a living standard. We show that the liquidity constraint endogenously determined becomes more tightened with a higher living standard. The optimal strategies with endogenous liquidity constraints are derived in closed form. We find a significant discontinuity and dramatic change in the effect of endogenous liquidity constraints on the optimal strategies, which in turn is determined by levels of current borrowing against future income. We show that consumption changes with respect to changes in wealth are greater with higher living standards when the amount of borrowing is large. However, this result can be reversed when the amount of borrowing is small. These findings are particularly important in addressing the interdependence of consumption and liquidity constraints to maintain a living standard in today's inflation crisis.}
	
		\href{chanwoolkim.github.io}{\textbf{Scale Up of an Influential Early Childhood Education Program}} \\ (with Andr\'es Hojman and Juan Pantano)
		
		Abstract: \textit{We compare the two similar yet distinct early childhood education (ECE) programs while addressing the concerns associated with comparing them. We revisit data from a series of randomized early childhood education interventions to investigate the effects of ECE participation at ages 0 to 3 on a child's cognitive outcome. We document treatment effect heterogeneity in ECE programs by drawing on insights from the causal forest algorithm following Athey and Wager (2018). In particular, children accrue different effects from participation in ECE programs, and the populations differ across programs. Hence, a natural question is what would be the effect of one if randomized into another population. Using a forest built on federal program data and applying state program data to obtain treatment effect estimates for a population resembling each other, we consider treatment heterogeneity and differences between the two program characteristics. The results suggest that when designed and targeted well, ECE programs may be a very effective tool to improve the lives of the disadvantaged population.}
	\end{rSection}
	
	%----------------------------------------------------------------------------------------
	%	WORK IN PROGRESS
	%----------------------------------------------------------------------------------------
	\begin{rSection}{Work in Progress}
		Balancing Household Debt and Municipal Revenues: Personalized Water Utility Pricing \\ (with Jean-Pierre Dub\'e and Sanjog Misra) \\
		Package Size Options and Unequal Burden of Inflation (with Youngeun Lee and Younggeun Yoo) \\
		Housing Tenure as an Investment: Evidence from Survey and Field Experiments (with Gieun Kim) \\
		Rational Addiction and Stimulant Prescription (with Giyoung Kwon and Younggeun Yoo)
	\end{rSection}

	%----------------------------------------------------------------------------------------
	%	AWARDS, SCHOLARSHIPS, AND GRANTS
	%----------------------------------------------------------------------------------------
	\begin{rSection}{Awards, Scholarships, and Grants}
		Henry Jr. Morgenthau Fellowship, University of Chicago \hfill {\em 2025} \\
		Agnes and Nathan Janco Travel Award, University of Chicago \hfill {\em 2024} \\
		Department Data Acquisition Grant, University of Chicago \hfill {\em 2024} \\
		Public Economics Initiative Research Grant, Becker Friedman Institute \hfill {\em 2024} \\
		Research Grant, Crazy Alpaca Inc. \hfill {\em 2023} \\
		Neubauer Fellowship, University of Chicago \hfill {\em 2018} \\
		Phi Beta Kappa, University of Chicago \hfill {\em 2017} \\
		General Honors, University of Chicago \hfill {\em 2017}
	\end{rSection}

	%----------------------------------------------------------------------------------------
	%	TEACHING EXPERIENCE
	%----------------------------------------------------------------------------------------
	\begin{rSection}{Teaching Experience}
		\begin{minipage}[t]{0.54\textwidth}
			Microeconomics (EMBA) \\
			Applied Industrial Organization (Undergraduate) \\
			Machine Learning for Economists (Masters) \\
			Elements of Economic Analysis II (Undergraduate) \\
			Price Theory I (PhD) \\
			Firm and the Non-Market Environment (MBA)
		\end{minipage}
		\begin{minipage}[t]{0.42\textwidth}
			TA for Lars Stole \hfill {\em Fall 2023, 2024} \\
			TA for Jonathan Arnold \hfill {\em Spring 2024} \\
			TA for Kirill Ponomarev \hfill {\em Winter 2024} \\
			College Lecturer \hfill {\em Winter 2023} \\
			TA for Kevin Murphy \hfill {\em Fall 2022} \\
			TA for Marianne Bertrand \hfill {\em Spring 2022}
		\end{minipage}
	\end{rSection}

	%----------------------------------------------------------------------------------------
	%	RESEARCH EXPERIENCE AND OTHER EMPLOYMENT
	%----------------------------------------------------------------------------------------
	\begin{rSection}{Research Experience and Other Employment}
		Research Assistant for Jean-Pierre Dub\'e, Booth School of Business \hfill {\em 2022-2024} \\
		Academic Consultant, Cracy Alpaca Inc. \hfill {\em 2022-2024} \\
		Intelligence and Operations Administrator (Sergeant), Republic of Korea Army \hfill {\em 2020-2022} \\
		Research Assistant for Kevin Murphy and John Huizinga, Booth School of Business \hfill {\em 2018} \\
		Research Professional for Pascal Noel and Peter Ganong, Booth School of Business \hfill {\em 2017-2018} \\
		Research Assistant for Casey Mulligan, Becker Friedman Institute \hfill {\em 2016-2018} \\
		Research Assistant for James Heckman and Juan Pantano, CEHD and NBER \hfill {\em 2015-2018}
	\end{rSection}
	
	%----------------------------------------------------------------------------------------
	%	PROFESSIONAL EXPERIENCE
	%----------------------------------------------------------------------------------------
	\begin{rSection}{Professional Experience}
		Organizer of Industrial Organization Lunch, University of Chicago \hfill {\em 2023-2025}
		% Member of Panel Y, University of Chicago \hfill {\em YYYY--YYYY}
		
		\begin{minipage}[t]{0.2\textwidth}
			\textbf{Conferences}
			
			\textbf{Presentations}
			
			% \textbf{Refereeing Activity}
		\end{minipage}
		\begin{minipage}[t]{0.8\textwidth}
			\textit{Advances with Field Experiments Conference (University of Chicago)}
			
			\textit{Economics Graduate Student Conference (Washington University in St. Louis)}
			
			\textit{College of Business (Korea Advanced Institute of Science \& Technology)}
			
			\textit{RUSH (Regional/Urban/Spatial/Housing) Graduate Student Brownbag}
			
			% \textit{Journal A}, \textit{Journal B}, \textit{Journal C}
		\end{minipage}
	
	\end{rSection}

	%----------------------------------------------------------------------------------------
	% 	OTHER WRITING
	%----------------------------------------------------------------------------------------
	% \begin{rSection}{Other Writing}
	% 	\href{www.personalwebsite.com/policy/pol1.pdf}{Title of Policy Article 1} (with First Last), \textit{Journal/Newspaper Name}, Other Info \\
	% 	\href{www.personalwebsite.com/policy/pol2.pdf}{Title of Policy Article 2} (with First Last), \textit{Journal/Newspaper Name}, Other Info
	% \end{rSection}

	%----------------------------------------------------------------------------------------
	%	ADDITIONAL INFORMATION
	%----------------------------------------------------------------------------------------
	\begin{rSection}{Additional Information}
		\begin{minipage}[t]{0.25\textwidth}
			\textbf{Citizenship} \\
			\textbf{Programming Skills} \\
			\textbf{Languages}
		\end{minipage}
		\begin{minipage}[t]{0.75\textwidth}
			Republic of Korea, U.S. Permanent Resident \\
			R, SQL, Python, Stata, Mathematica, MATLAB \\
			English (fluent), Korean (native), French (intermediate)
		\end{minipage}
	\end{rSection}

	\vfill
	\flushright \textit{This version: \today}
\end{document}
